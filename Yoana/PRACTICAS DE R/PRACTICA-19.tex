\documentclass[12pt,letterpaper]{article}\usepackage[]{graphicx}\usepackage[]{color}
%% maxwidth is the original width if it is less than linewidth
%% otherwise use linewidth (to make sure the graphics do not exceed the margin)
\makeatletter
\def\maxwidth{ %
  \ifdim\Gin@nat@width>\linewidth
    \linewidth
  \else
    \Gin@nat@width
  \fi
}
\makeatother

\definecolor{fgcolor}{rgb}{0.345, 0.345, 0.345}
\newcommand{\hlnum}[1]{\textcolor[rgb]{0.686,0.059,0.569}{#1}}%
\newcommand{\hlstr}[1]{\textcolor[rgb]{0.192,0.494,0.8}{#1}}%
\newcommand{\hlcom}[1]{\textcolor[rgb]{0.678,0.584,0.686}{\textit{#1}}}%
\newcommand{\hlopt}[1]{\textcolor[rgb]{0,0,0}{#1}}%
\newcommand{\hlstd}[1]{\textcolor[rgb]{0.345,0.345,0.345}{#1}}%
\newcommand{\hlkwa}[1]{\textcolor[rgb]{0.161,0.373,0.58}{\textbf{#1}}}%
\newcommand{\hlkwb}[1]{\textcolor[rgb]{0.69,0.353,0.396}{#1}}%
\newcommand{\hlkwc}[1]{\textcolor[rgb]{0.333,0.667,0.333}{#1}}%
\newcommand{\hlkwd}[1]{\textcolor[rgb]{0.737,0.353,0.396}{\textbf{#1}}}%

\usepackage{framed}
\makeatletter
\newenvironment{kframe}{%
 \def\at@end@of@kframe{}%
 \ifinner\ifhmode%
  \def\at@end@of@kframe{\end{minipage}}%
  \begin{minipage}{\columnwidth}%
 \fi\fi%
 \def\FrameCommand##1{\hskip\@totalleftmargin \hskip-\fboxsep
 \colorbox{shadecolor}{##1}\hskip-\fboxsep
     % There is no \\@totalrightmargin, so:
     \hskip-\linewidth \hskip-\@totalleftmargin \hskip\columnwidth}%
 \MakeFramed {\advance\hsize-\width
   \@totalleftmargin\z@ \linewidth\hsize
   \@setminipage}}%
 {\par\unskip\endMakeFramed%
 \at@end@of@kframe}
\makeatother

\definecolor{shadecolor}{rgb}{.97, .97, .97}
\definecolor{messagecolor}{rgb}{0, 0, 0}
\definecolor{warningcolor}{rgb}{1, 0, 1}
\definecolor{errorcolor}{rgb}{1, 0, 0}
\newenvironment{knitrout}{}{} % an empty environment to be redefined in TeX

\usepackage{alltt}
 \usepackage[left=2cm,right=2cm,top=2cm,bottom=2cm]{geometry}
\usepackage[ansinew]{inputenc}
\usepackage[spanish]{babel}
\usepackage{amsmath}
\usepackage{amsfonts}
\usepackage{amssymb}
\usepackage{dsfont}
\usepackage{multicol} 
\usepackage{subfigure}
\usepackage{graphicx}
\usepackage{float} 
\usepackage{verbatim} 
\usepackage[left=2cm,right=2cm,top=2cm,bottom=2cm]{geometry}
\usepackage{fancyhdr}
\pagestyle{fancy} 
\fancyhead[LO]{\leftmark}
\usepackage{caption}
\newtheorem{definicion}{Definci\'on}
\IfFileExists{upquote.sty}{\usepackage{upquote}}{}
\begin{document}

\begin{titlepage}
\setlength{\unitlength}{1 cm} %Especificar unidad de trabajo


\begin{center}
\textbf{{\large UNIVERSIDAD DE EL SALVADOR}\\
{\large FACULTAD MULTIDISCIPLINARIA DE OCCIDENTE}\\
{\large DEPARTAMENTO DE MATEM\'ATICA}}\\[0.50 cm]

\begin{picture}(18,4)
 \put(7,0){\includegraphics[width=4cm]{minerva.jpg}}
\end{picture}
\\[0.25 cm]

\textbf{{\large Licenciatura en Estad\'istica}\\[1.25cm]
{\large Control Estadistico del Paquete R }\\[2 cm]
%\setlength{\unitlength}{1 cm}
{\large  \textbf{''UNIDAD CUATRO"}}\\
{\large  \textbf{Pr\'actica 19 - Estimaci\'on por intervalos de confianza. Dos poblaciones}}\\[3 cm]
{\large Alumna:}\\
{\large Martha Yoana Medina S\'anchez}\\[2cm]
{\large Fecha de elaboraci\'on}\\
Santa Ana - \today }
\end{center}
\end{titlepage}

\newtheorem{teorema}{Teorema}
\newtheorem{prop}{Proposici\'on}[section]

\lhead{UNIDAD CUATRO}
\chead{PR\'ACTICA 19}
\lfoot{LICENCIATURA EN ESTAD\'ISTICA}
\cfoot{UESOCC}
\rfoot{\thepage}
%\pagestyle{fancy} 

\setcounter{page}{1}
\newpage

\textbf{Estimaci\'on por intervalos de confianza. Dos poblaciones}\\

\textbf{Ejercicio 1.}\\

\begin{knitrout}
\definecolor{shadecolor}{rgb}{0.969, 0.969, 0.969}\color{fgcolor}\begin{kframe}
\begin{alltt}
\hlstd{Cadiz} \hlkwb{<-} \hlkwd{c} \hlstd{(}\hlnum{182}\hlstd{,} \hlnum{170}\hlstd{,} \hlnum{175}\hlstd{,} \hlnum{167}\hlstd{,} \hlnum{171}\hlstd{,}  \hlnum{174}\hlstd{,}  \hlnum{181}\hlstd{,}  \hlnum{169}\hlstd{,}  \hlnum{174}\hlstd{,}  \hlnum{174}\hlstd{,}  \hlnum{170}\hlstd{,}  \hlnum{176}\hlstd{,} \hlnum{168}\hlstd{,} \hlnum{178}\hlstd{,}  \hlnum{180}\hlstd{);}
\hlstd{(}\hlkwd{var}\hlstd{(Cadiz))}
\end{alltt}
\begin{verbatim}
## [1] 22.92381
\end{verbatim}
\begin{alltt}
\hlstd{Malaga} \hlkwb{<-} \hlkwd{c} \hlstd{(}\hlnum{181}\hlstd{,} \hlnum{173}\hlstd{,} \hlnum{177}\hlstd{,} \hlnum{170}\hlstd{,} \hlnum{170}\hlstd{,}  \hlnum{175}\hlstd{,}  \hlnum{169}\hlstd{,}  \hlnum{169}\hlstd{,}  \hlnum{171}\hlstd{,}  \hlnum{173}\hlstd{,}  \hlnum{177}\hlstd{,}  \hlnum{182}\hlstd{,} \hlnum{179}\hlstd{,} \hlnum{165}\hlstd{,}  \hlnum{174}\hlstd{);}
\hlstd{(}\hlkwd{var}\hlstd{(Malaga))}
\end{alltt}
\begin{verbatim}
## [1] 23.52381
\end{verbatim}
\begin{alltt}
\hlcom{# VARIANZAS DESCONOCIDAS PERO DIFERENTES (MUESTRAS PEQUE\textbackslash{}~NAS)}

\hlcom{# Creando nuestra propia funci\textbackslash{}'on}

\hlstd{intervalovardesigual} \hlkwb{<-} \hlkwa{function}\hlstd{(}\hlkwc{n1}\hlstd{,} \hlkwc{n2}\hlstd{,} \hlkwc{nivel.conf}\hlstd{=}\hlnum{0.95}\hlstd{)}

\hlstd{\{}
  \hlstd{varianza_1} \hlkwb{=} \hlstd{(}\hlkwd{sd}\hlstd{(Cadiz))}
  \hlstd{varianza_2} \hlkwb{=} \hlstd{(}\hlkwd{sd}\hlstd{(Malaga))}
  \hlstd{Media_1} \hlkwb{=} \hlstd{(}\hlkwd{mu}\hlstd{(Candiz))}
  \hlstd{Media_2} \hlkwb{=} \hlstd{(}\hlkwd{mu}\hlstd{(Malaga))}
  \hlstd{m_1} \hlkwb{=} \hlstd{n1} \hlopt{-} \hlnum{1}
  \hlstd{m_2} \hlkwb{=} \hlstd{n2} \hlopt{-} \hlnum{1}
  \hlstd{w} \hlkwb{=} \hlstd{((varianza_1}\hlopt{/}\hlstd{n1)} \hlopt{+} \hlstd{(varianza_2}\hlopt{/}\hlstd{n2))}\hlopt{^}\hlnum{2}



\hlstd{\}}
\end{alltt}
\end{kframe}
\end{knitrout}


\end{document}
