\documentclass[12pt,letterpaper]{article}\usepackage[]{graphicx}\usepackage[]{color}
%% maxwidth is the original width if it is less than linewidth
%% otherwise use linewidth (to make sure the graphics do not exceed the margin)
\makeatletter
\def\maxwidth{ %
  \ifdim\Gin@nat@width>\linewidth
    \linewidth
  \else
    \Gin@nat@width
  \fi
}
\makeatother

\definecolor{fgcolor}{rgb}{0.345, 0.345, 0.345}
\newcommand{\hlnum}[1]{\textcolor[rgb]{0.686,0.059,0.569}{#1}}%
\newcommand{\hlstr}[1]{\textcolor[rgb]{0.192,0.494,0.8}{#1}}%
\newcommand{\hlcom}[1]{\textcolor[rgb]{0.678,0.584,0.686}{\textit{#1}}}%
\newcommand{\hlopt}[1]{\textcolor[rgb]{0,0,0}{#1}}%
\newcommand{\hlstd}[1]{\textcolor[rgb]{0.345,0.345,0.345}{#1}}%
\newcommand{\hlkwa}[1]{\textcolor[rgb]{0.161,0.373,0.58}{\textbf{#1}}}%
\newcommand{\hlkwb}[1]{\textcolor[rgb]{0.69,0.353,0.396}{#1}}%
\newcommand{\hlkwc}[1]{\textcolor[rgb]{0.333,0.667,0.333}{#1}}%
\newcommand{\hlkwd}[1]{\textcolor[rgb]{0.737,0.353,0.396}{\textbf{#1}}}%

\usepackage{framed}
\makeatletter
\newenvironment{kframe}{%
 \def\at@end@of@kframe{}%
 \ifinner\ifhmode%
  \def\at@end@of@kframe{\end{minipage}}%
  \begin{minipage}{\columnwidth}%
 \fi\fi%
 \def\FrameCommand##1{\hskip\@totalleftmargin \hskip-\fboxsep
 \colorbox{shadecolor}{##1}\hskip-\fboxsep
     % There is no \\@totalrightmargin, so:
     \hskip-\linewidth \hskip-\@totalleftmargin \hskip\columnwidth}%
 \MakeFramed {\advance\hsize-\width
   \@totalleftmargin\z@ \linewidth\hsize
   \@setminipage}}%
 {\par\unskip\endMakeFramed%
 \at@end@of@kframe}
\makeatother

\definecolor{shadecolor}{rgb}{.97, .97, .97}
\definecolor{messagecolor}{rgb}{0, 0, 0}
\definecolor{warningcolor}{rgb}{1, 0, 1}
\definecolor{errorcolor}{rgb}{1, 0, 0}
\newenvironment{knitrout}{}{} % an empty environment to be redefined in TeX

\usepackage{alltt}
 \usepackage[left=2cm,right=2cm,top=2cm,bottom=2cm]{geometry}
\usepackage[ansinew]{inputenc}
\usepackage[spanish]{babel}
\usepackage{amsmath}
\usepackage{amsfonts}
\usepackage{amssymb}
\usepackage{dsfont}
\usepackage{multicol} 
\usepackage{subfigure}
\usepackage{graphicx}
\usepackage{float} 
\usepackage{verbatim} 
\usepackage[left=2cm,right=2cm,top=2cm,bottom=2cm]{geometry}
\usepackage{fancyhdr}
\pagestyle{fancy} 
\fancyhead[LO]{\leftmark}
\usepackage{caption}
\newtheorem{definicion}{Definción}
\IfFileExists{upquote.sty}{\usepackage{upquote}}{}
\begin{document}

\begin{titlepage}
\setlength{\unitlength}{1 cm} %Especificar unidad de trabajo

\begin{center}
\textbf{{\large UNIVERSIDAD DE EL SALVADOR}\\
{\large FACULTAD MULTIDISCIPLINARIA DE OCCIDENTE}\\
{\large DEPARTAMENTO DE MATEM\'ATICA}}\\[0.50 cm]

\begin{picture}(18,4)
 \put(7,0){\includegraphics[width=4cm]{minerva.jpg}}
\end{picture}
\\[0.25 cm]

\textbf{{\large Licenciatura en Estad\'istica}\\[1.25cm]
{\large Control Estad\'istico del Paquete R }\\[2 cm]
%\setlength{\unitlength}{1 cm}
{\large  \textbf{''UNIDAD UNO"}}\\
{\large  \textbf{PR\'ACTICA 04 - Tipos de objetos, operadores y funciones que operan sobre ellos: Vectores, matrices y arreglos (matrices indexadas). .}}\\[3 cm]
{\large Alumna:}\\
{\large Martha Yoana Medina Sanch\'ez}\\[2cm]
{\large Fecha de elaboraci\'on}\\
Santa Ana - \today }
\end{center}
\end{titlepage}

\newtheorem{teorema}{Teorema}
\newtheorem{prop}{Proposici?n}[section]

\lhead{Pr\'actica 04}

\lfoot{LICENCIATURA EN ESTAD\'iSTICA}
\cfoot{UESOCC}
\rfoot{\thepage}
%\pagestyle{fancy} 

\setcounter{page}{1}
\newpage

\section{IMPORTACI\'ON Y EXPORTACI\'ON DE DATOS EN R}

Generalmente los datos suelen leerse desde archivos externos y no teclearse desde la consola. Las
capacidades de lectura de archivos de R son sencillas y sus requisitos son bastante estrictos, por lo
que hay que tenerlas muy en cuenta, de lo contrario los resultados en la lectura no ser\'an los
esperados.
\subsection{USO DE LA FUNCI\'ON READ.TABLE()}
\begin{itemize}
\item Ejemplo: Guardar (escribir) determinados datos en un archivo de texto (ASCII) y luego recuperar
(leer) dicho archivo desde R.
\end{itemize}
\begin{itemize}
\item {1) } Cambiar el directorio de trabajo a su directorio de trabajo, en el cual ha almacenado sus
pr\'acticas, desde el men\'u File.\\
\item {2) } Abrir el R Editor para crear un nuevo script desde el men\'u File.\\
\item {3) } En la ventana del R Editor, teclee los datos tal como se muestra:\\
\end{itemize}
Observaciones:
\begin{itemize}
\item La primera l\'inea del archivo debe contener el nombre de cada objeto o variable.
\item En cada una de las siguientes l\'ineas, el primer elemento es la etiqueta de la fila, y a
continuaci\'on deben aparecer los valores de cada variable.
\item Si el archivo tiene un elemento menos en la primera l\'inea que en las restantes,
obligatoriamente ser\'a el dise\~no anterior el que se utilice.
\item A menudo no se dispone de etiquetas de filas. En ese caso, tambi\'en es posible la lectura y el programa a\~nadir\'i unas etiquetas predeterminadas.
\item La \'ultima l\'inea debe finalizar con ENTER para que R reconozca el fin del archivo.
\end{itemize}

\begin{knitrout}
\definecolor{shadecolor}{rgb}{0.969, 0.969, 0.969}\color{fgcolor}\begin{kframe}
\begin{alltt}
\hlstd{Entrada1} \hlkwb{<-} \hlkwd{read.table}\hlstd{(}\hlstr{"datos01.txt"}\hlstd{,} \hlkwc{header}\hlstd{=T);Entrada1}
\end{alltt}


{\ttfamily\noindent\color{warningcolor}{\#\# Warning in read.table("{}datos01.txt"{}, header = T): incomplete final line found by readTableHeader on 'datos01.txt'}}\begin{verbatim}
##   EDAD ESTATURA PESO SEXO
## 1   26     1.65  146    F
## 2   21     1.73  158    M
## 3   21     1.81  167    M
## 4   20     1.70  152    F
\end{verbatim}
\begin{alltt}
\hlstd{Entrada2} \hlkwb{<-} \hlkwd{read.table}\hlstd{(}\hlstr{"datos01.dat"}\hlstd{,} \hlkwc{header}\hlstd{=T);Entrada2}
\end{alltt}


{\ttfamily\noindent\color{warningcolor}{\#\# Warning in file(file, "{}rt"{}): no fue posible abrir el archivo 'datos01.dat': No such file or directory}}

{\ttfamily\noindent\bfseries\color{errorcolor}{\#\# Error in file(file, "{}rt"{}): no se puede abrir la conexión}}

{\ttfamily\noindent\bfseries\color{errorcolor}{\#\# Error in eval(expr, envir, enclos): objeto 'Entrada2' no encontrado}}\begin{alltt}
\hlcom{# No existe diferencia entre ambos archivos a la hora de leerlos }
\end{alltt}
\end{kframe}
\end{knitrout}







\end{document}
